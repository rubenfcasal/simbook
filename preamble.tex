\usepackage[default]{fontsetup}
\usepackage{booktabs}
\usepackage{lmodern}
\usepackage[a4paper, top=3.25cm, bottom=2.5cm, inner=2.5cm, outer=2.5cm]{geometry}
%\usepackage[papersize={19cm, 24cm}, top=2.5cm, bottom=2cm, inner=2.5cm, outer=2cm]{geometry}
%\usepackage{animate}
%\usepackage{fontspec}
%\setmainfont{Arial}
% ---------------------------
\usepackage{emptypage}
% líneas viudas y huérfanas
\widowpenalty=500
\clubpenalty=500
% ---------------------------
% Space above/below an [h] float (default 12pt)
\setlength{\intextsep}{18pt}
% Leyendas de figuras y tablas
\usepackage{caption}
\captionsetup{font=small, labelfont=bf, margin={.08\textwidth}}
%\captionsetup{font=small, labelfont=bf, margin={.08\textwidth, .02\textwidth}}
% ---------------------------
% Espacio después de teorema
% https://tex.stackexchange.com/questions/37797/theorem-environment-line-break-after-label
% https://tex.stackexchange.com/questions/8110/is-it-possible-to-skip-the-first-line-in-a-theorem-environment
\usepackage{ntheorem}
\theoremstyle{break}
\theoremheaderfont{\normalfont\bfseries}
\theorembodyfont{\normalfont}
\theorempreskip{\bigskipamount}
\theorempostskip{\smallskipamount}
\theoremprework{\bigskip\hrule\leavevmode}
\theoremseparator{\smallskip}
\theorempostwork{\bigskip\hrule\bigskip}
\newtheorem{theorem}{Teorema}[chapter]
\theoremprework{\bigskip\hrule\leavevmode}
\theorempostwork{\bigskip\hrule\bigskip}
\newtheorem{conjecture}{Algoritmo}[chapter]
\newtheorem{lemma}{Lema}[chapter]
\newtheorem{corollary}{Corolario}[chapter]
\newtheorem{proposition}{Proposición}[chapter]
\newtheorem{definition}{Definición}[chapter]
\newtheorem{hypothesis}{Hipótesis}[chapter]
\newtheorem{exercise}{Ejercicio}[chapter]
%\theoremprework{\bigskip\leavevmode}
%\theorempostwork{\vspace*{-\bigskipamount}\vspace*{-\medskipamount}}
\newtheorem{example}{Ejemplo}[chapter]
\theoremstyle{nonumberplain}
\theoremheaderfont{\normalfont\itshape}
\theoremseparator{:}
% \theorempostskip{\bigskipamount} % PENDIENTE
% \theorempostwork{\hrule}
\newtheorem{remark}{Nota}
\newtheorem{solution}{Solución}
\newtheorem{proof}{Demostración}
% ---------------------------
% Spanish
\ifxetex
  \usepackage{polyglossia}
  \setmainlanguage{spanish}
  % Tabla en lugar de cuadro
  \gappto\captionsspanish{\renewcommand{\tablename}{Tabla}
          \renewcommand{\listtablename}{Índice de tablas}}
\else
  \usepackage[spanish,es-tabla]{babel}
\fi
\usepackage{csquotes}
% Añadir bibliografía en el índice
\usepackage[nottoc,notlot,notlof]{tocbibind}
\addto\captionsspanish{\renewcommand{\contentsname}{\'Indice}}
% ---------------------------
% Encabezados
\usepackage{fancyhdr}
\renewcommand{\chaptermark}[1]{\markboth{\thechapter.\ #1}{}}
\renewcommand{\sectionmark}[1]{\markright{\thesection.\ #1}}
\pagestyle{fancy}
\fancyhf{}
\fancyhead[LE,RO]{\thepage}
\fancyhead[LO]{\itshape\nouppercase{\rightmark}}
\fancyhead[RE]{\itshape\nouppercase{\leftmark}}
\renewcommand{\headrulewidth}{0pt}
\setlength{\headheight}{13.2pt}
% ---------------------------
% Code chunks
% https://github.com/tidymodels/TMwR/blob/main/latex_extras/preamble.tex
%\DefineVerbatimEnvironment{Highlighting}{Verbatim}{commandchars=\\\{\},fontsize=\small}
%\DefineVerbatimEnvironment{Highlighting}{Verbatim}{commandchars=\\\{\},fontsize=\small, xleftmargin=2mm}
\DefineVerbatimEnvironment{Highlighting}{Verbatim}{commandchars=\\\{\}, xleftmargin=2mm}
\renewcommand{\CommentTok}[1]{\textcolor[rgb]{0.41,0.41,0.41}{\texttt{#1}}}
% https://tex.stackexchange.com/questions/161678/decreasing-font-size-in-verbatim
% https://tex.stackexchange.com/questions/56970/how-can-i-set-stretch-of-verbatim-to-be-different-than-global-setting
\usepackage[nodisplayskipstretch]{setspace} \setstretch{1.2}
%\usepackage{etoolbox}
% Interlineado code chunk
\preto{\Shaded}{\vspace{0.2\baselineskip}\edef\tempstretch{\baselinestretch}\par\setstretch{1}}
\appto{\endShaded}{\vspace{-\tempstretch\baselineskip}\vspace{1.2\baselineskip}}
% Interlineado resultados y tamaño
%\preto{\verbatim}{\edef\tempstretch{\baselinestretch}\par\setstretch{1}\small}
%\appto{\endverbatim}{\vspace{-\tempstretch\baselineskip}\normalsize\vspace{1.2\baselineskip}}
\preto{\verbatim}{\edef\tempstretch{\baselinestretch}\par\setstretch{1}}
\appto{\endverbatim}{\vspace{-\tempstretch\baselineskip}\vspace{1.2\baselineskip}}
% ---------------------------
% Notas al pie
\let\oldfootnote\footnote
\renewcommand\footnote[1]{\oldfootnote{\hspace{2mm}#1}}
% ---------------------------
\usepackage{microtype}
\usepackage{hyperref}
